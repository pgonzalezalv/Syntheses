\documentclass[fr]{../../../../../../eplexam}

\hypertitle{Mathématiques discrètes et probabilité}{3}{EPL}{1108}{2020}{Janvier}{All}
{Baptiste Laterre}
{Jean-Charles Delvenne et Olivier Pereira}

\section{Question 1}
Soit un graphe pondéré donné. Trouvez le coût du plus court chemin pour aller de $s$ à $d$.
\nosolution
\section{Question 2}
Trouvez le résultat de $2^{2020}$ \% 23
\nosolution
\section{Question 3}
Vous vous tenez à proximité d'une bombe à retardement : celle-ci affiche un décompte de 22 minutes. Vous appercevez 5 câbles et allez en sectionner afin d'essayer de désamorcer la bombe. Sachant que la découpe de chaque câble vous prend 7 minutes, vous ne pouvez en choisir que 3. Parmi les 5 câbles, la section de 3 ne provoque rien, 1 vous permet de désamorcer la bombe et le dernier fait exploser celle-ci. Quelle est la probabilité que vous parveniez à désamorcer la bombe ? 
\nosolution
\section{Question 4}
Vous avez dans votre poche 3 pièces de monnaies, dont 2 identiques et une truquée : pour cette dernière, la probabilité que vous tombiez sur pile en la lançant est de 0.75, alors que celle des 2 autres vaut 0.5. Sachant que le choix d'une pièce dans votre poche est équiprobable, sachant que vous obtenez pile en lançant une pièce, quelle est la probabilité que vous ayez sélectionné la pièce truquée ?
\nosolution
\section{Question 5}
Soient 2 variables aléatoires : A = $X$ et B = $X^2$. Calculez le carré de la corrélation de ces 2 variables.
\nosolution
\section{Question 6}
A partir d'une suite de chiffre allant de 1 à 5 (compris), trouvez la probabilité que cette suite soit telle qu'elle soit constamment croissante, sauf en un endroit(par exemple, 3,4,1,2,5). Cette suite a une probabilité de 1/120 d'être constamment croissante (1,2,3,4,5).
\nosolution
\section{Question 7}
Vous réalisez l'expérience du dé (non-truqué : la probabilité d'obtenir une certaine face est de 1/6). Après avoir réalisé N lancés, vous obtenez K fois la face 6. Calculez alors les limites suivantes : 
\begin{itemize}
    \item $\lim_{N \to \infty} P(K/N = 1/6)$ =
    \item $\lim_{N \to \infty} P(0.1 \leq K/N \leq 0.2$ = 
    \item $\lim_{N \to \infty} P(K \leq N/6 + \sqrt{5N}/3$ =
\end{itemize}
\section{Question 8}
Concernait l'ordre du groupe quotient G/H, en devant trouver |H| connaissant G.
\section{Question 9}
Alice et Bob se rendent à la poste, entre 15 et 16h. Leur arrivée est indépendante et équiprobable. Pour simplifier le calcul, on considérera des temps entre 0 et 1 : 0 = 15h, 1 = 16h, ¼ = 15h15, ½ = 15h30, … Soit $T_A$ et $T_B$ les variables aléatoires des heures d’arrivées de Alice et Bob à la poste. Nous définirons $T_{min}$ = $min (T_A,T_B)$ et $T_{max}$ = $max (T_A,T_B)$. 
\begin{enumerate}
    \item Donnez l’expression de $F_{min}$, la fonction de répartition de $T_{min}$. Dessinez-la également.
    \item Donnez l’expression de $f_{min}$, la densité de probabilité de $T_{min}$. Dessinez-la également.
    \item Trouvez l'espérance de $T_{min}$.
    \item BONUS : Trouvez la densité de probabilité de de $T_{max}$ - $T_{min}$ en sachant que $T_{min}$ = s et sans connaître $T_{max}$.
\end{enumerate}
Indication : si $T_{min}$ < ¼, la probabilité que Alice ou Bob arrive avant 15.15 est nulle.
\nosolution
\section{Question 10}
Soit un graphe orienté et sans cycle G := (N,R). Démontrez que ce graphe possède une source, c-à-d un nœud i tel que $\forall$ j $\in$ N, (j,i) $\notin$ R.
\nosolution
\end{document}
