\documentclass[fr]{../../../../../../eplexam}

\usepackage{enumitem}

\hypertitle{Mécanique des sols}{5}{GCIV}{1072}{2018}{Juin}{Mineure}
{Aquin Magnus}
{Benoît Pardoen}

\section{}
\begin{enumerate}[label=(\alph*)]
\item Définissez l’état critique du sol
\item Illustrez le comportement des échantillons de sol, denses et lâches soumis au cisaillement. Quel est l’impact de la densité initiale sur la résistance au cisaillement à l’état critique ?
\item	Définissez la ligne d’état critique et le critère de résistance au cisaillement en tenant compte des effets de densité. 
\item	Illustrez les chemins de contrainte (p’,q) et (p’,v) des échantillons soumis à un cisaillement non drainé (triaxial). Mentionnez les conditions d’essai. Détaillez l’effet de la densité initiale et du degré de consolidation sur les chemins de contrainte.


\end{enumerate}


\section{}
\begin{enumerate}[label=(\alph*)]
\item Illustrez le mécanisme de rupture d’une fondation superficielle filante sur sol pulvérulent selon Prandtl. Mentionnez les hypothèses nécessaires. 
\item	Définissez les différentes zones et lignes de rupture qui composent le schéma cinématique global de rupture.
\item Pour chaque zone, détaillez leur état, les contraintes en jeu et les cercles de Mohr à la rupture.
\item	Définissez la contribution au pouvoir portant ultime par le modèle de Prandtl. Démontrez comment obtenir ce terme et à partir de quel équilibre.
\item	Définissez la formule canonique du pouvoir portant. Expliquez comment sont obtenus les autres termes. Par quelles caractérisations du sol la charge ultime est-elle influencée ?
\item	Expliquez brièvement comment réaliser un dimensionnement à la rupture
an apple
\end{enumerate}
\section{}Définissez
\begin{enumerate}[label=(\alph*)]

    \item Degré de consolidation
\item	$C_c, C_s, m_v, T_v$
\item	Tassement séculaire
\item	Différence entre pieux vissés et forés

\end{enumerate}

\end{document}
