\documentclass[fr]{../../../../../../eplexam}

\usepackage{../../../mmc-MECA1901-exam}

\hypertitle{Mécanique des milieux continus}{5}{MECA}{1901}{2015}{Janvier}{Mineure}
{Vincent Schellekens\and Antoine de Comité\and Aurélien Pignolet\and Mamadou Segpa\and Philippe Greiner}
{Philippe Chatelain et Issam Doghri}

\section{Théorie}

\begin{enumerate}
  \item Prouver en utilisant la notation indicielle que $\nabla \wedge \nabla \alpha = 0$ pour un champ scalaire $\alpha$

\begin{solution}
Réecrivons l'expression en notation indicielle:
\begin{align}
\nabla \wedge \nabla \alpha & =\left(\bm{\hat{e_j}}\frac{\partial}{\partial x_j}\right) \wedge \left(\frac{\partial a}{\partial x_i}\bm{\hat{e_i}}\right)\\
& = \epsilon_{ijk}\bm{\hat{e_k}}\frac{\partial^2 a}{\partial x_i \partial x_j}\\
\end{align}

On peut écrire, pour la $k^{\text{ième}}$ composante du vecteur:
\begin{align*}
\epsilon_{ijk}\frac{\partial^2 a}{\partial x_i \partial x_j} & = -\epsilon_{jik}\frac{\partial^2 a}{\partial x_i \partial x_j}\\
 & = -\epsilon_{ijk}\frac{\partial^2 a}{\partial x_j \partial x_i} \\
 & = -\epsilon_{ijk}\frac{\partial^2 a}{\partial x_i \partial x_j}
\end{align*}
Pour écrire la première ligne, nous utilisons la définition du symbôle de \textit{levi-civata}. La seconde ligne est obtenue en renommant i, j et inversément. La dernière ligne est obtenue grâce à la symétrie de $\epsilon_{ijk}\bm{\hat{e_k}}\frac{\partial^2 a}{\partial x_i \partial x_j}$.

On remarque que l'expression est égale à son opposée et ceci n'est vrai que dans le cas où elle est nulle. On en conclut donc que $\nabla \wedge \nabla \alpha = 0$.
\end{solution}

\item
Interpréter le terme $D_{12}$

\begin{solution}
La définition de $D_{12}$ en coordonnées cartésiennes est donnée par:
\begin{equation*}
D_{12}=D_{21}=\frac{1}{2}\left(\frac{\partial v_1}{\partial x_2}+\frac{\partial v_2}{\partial x_1}\right)
\end{equation*}
Il s'agit du taux de déformation associé aux directions $e_1$, $e_2$.
\end{solution}

\item
Soit un milieu en contrainte bi-axiale : $\sigma_1>\sigma_2>0$. Dessiner le cercle de Mohr et donner la contrainte tangentielle max et la facette ou celle-ci est la plus grande

\begin{solution}
Dans un milieu en contrainte bi-axiale, il y a deux contraintes principales non nulles et une contrainte principale nulle. Pour trouver la contrainte tangentielle maximale, il faut se promener sur le cercle de Mohr extérieur, celui qui représente le plan formé par les directions principales associées à $\sigma_1$ et 0. La contrainte tangentielle maximale est obtenue lorsqu'on a parcouru un angle de 90 degrés. Ce qui correspond, dans le problème à la bissectrice entre les deux directions principales qui forment ce plan. La contrainte tangentielle maximale est donnée par $\frac{\sigma_1}{2}$.
\end{solution}

\item
Expliquer le théorème de Green-Naghdi-Rivlin. Enoncer brièvement comment le prouver.

\begin{solution}
Le théorème de Green-Naghdi-Rivlin stipule que, à partir des l'expression de la conservation de l'énergie et du principe d'invariance pour les mouvements rigides, on peut dériver les expressions de la conservation de la masse, de la quantité de mouvement et du moment de la quantité de mouvement. Le principe d'invariance pour les mouvements rigides stipule que l'expression de la conservation de l'énergie est encore vérifiée même si le milieu est soumis à un mouvement rigide (translation, rotation).

Pour le démontrer, il faut considérer séparément un mouvement de translation simple et de rotation rigide. Le principe d'invariance nous permet de réecrire l'équation de la conservation de l'énergie en tenant compte des changements apportés. Dans le cas de la translation, seule l'expression de la vitesse est modifiée tandis que dans le cas de la rotation rigide, la vitesse et les forces à distances sont modifiées (ajout de Coriolis et de la force centrifuge qui sont des forces fictives).
\end{solution}

\item
Montrer comment on peut simplifier la loi de constitution d'un fluide visqueux Newtonien dans le cas incompressible.

\begin{solution}
La loi de constitution d'un fluide visqueux Newtonien est donnée par:
\begin{equation*}
\sigma=2\mu \cal{D}+\lambda (\text{tr}\cal{D})\cal{I}-\text{p}\cal{I}
\end{equation*}
Si on a affaire à un fluide incompressible, on peut dire que la masse volumique ne varie pas en fonction du temps et donc, si on reprend l'équation de la conservation de la masse donnée par:
\begin{equation*}
\frac{D\rho}{Dt}+\rho \nabla\cdot \bm{v}=0
\end{equation*}
On trouve que $\nabla\cdot \bm{v}=0$, et donc on peut dire que la trace du tenseur des taux de déformation est nulle
\begin{equation*}
\sigma=2\mu \cal{D}-\text{p}\cal{I}
\end{equation*}
\end{solution}
\end{enumerate}

\end{document}
