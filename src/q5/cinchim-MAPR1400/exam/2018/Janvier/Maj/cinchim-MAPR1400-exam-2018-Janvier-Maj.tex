\documentclass[fr]{../../../../../../eplexam}

\hypertitle{Cinétique chimique}{5}{MAPR}{1400}{2018}{Janvier}{Majeure}
{Thibaut Heremans}
{Juray de Wilde et Christian Bailly}

\paragraph{Informations générales} 2h Bailly, pause 10 min, 2h Juray, on pouvait commencer l'exo qd on voulait (30min).


\section*{C. Bailly}
\paragraph{Question 1}
\begin{itemize}
\item Comment peut-on comprendre une réaction unimoléculaire dans le cadre de la théorie des collisions?
\item Comparer avec la théorie du complexe activée.
\end{itemize}

\paragraph{Question 2}
\begin{itemize}
\item Faire la distinction entre la notion de chaine cinétique et de chaine matérielle dans le cas d'une polymérisation radicalaire.
\item Comment l'ajout d'un agent de transfert influe-t-il la longueur des deux chaines?
\end{itemize}

\paragraph{Question 3}
\begin{itemize}
\item Que signifie la distribution de longueur des chaînes, dans le cas d'une polymérisation radicalaire et terminaison par dismutation?
\end{itemize}


\paragraph{Question 4}
\begin{itemize}
\item Comment obtient-on l'équation de la diffusion (seconde loi de Fick) pour un problème à une dimension. Interprétez ce résultat d'un point de vue statistique (pas de détails).
\item Que devient cette loi lorsqu'on tient compte d'une réaction? Montrer comment rendre cette expression adimentionnelle.
\end{itemize}



\paragraph{Exercice} On considère la réaction entre l'$H_2$ et le $Cl_2$ en présence d'oxygène. 

$$ Cl_2\overset{r_i}{\xrightarrow{\hspace*{1cm}}}  2Cl^{\bullet}$$
$$ Cl^{\bullet} + H_2\overset{k_1}{\xrightarrow{\hspace*{1cm}}}  HCl + H^{\bullet}$$
$$ H^{\bullet}+Cl_2\overset{k_2}{\xrightarrow{\hspace*{1cm}}}  HCl+Cl^{\bullet}$$
$$ Cl^{\bullet} + O_2 \overset{k_3}{\xrightarrow{\hspace*{1cm}}} \textup{produits inactifs} $$
$$ H^{\bullet} + O_2 \overset{k_4}{\xrightarrow{\hspace*{1cm}}} \textup{produits inactifs}$$





\begin{itemize}
\item Écrire le nom des étapes.
\item Est-ce une réaction radicalaire en chaine droite ou ramifiée?
\item En faisant l'hypothèse de quasi-stationnarité des radicaux, obtenir une expression des concentration de $H^{\bullet}$ et le $Cl^{\bullet}$.
\item En déduire la vitesse de disparition des réactifs.
\item En faisant l'hypothèse des chaines longues, quelles seraient les différences?
\end{itemize}


\section*{J. De Wilde}
\paragraph{Question 1}
\begin{itemize}
\item Montrer comment les limitations de transfert de masse interfaciales peuvent influencer les vitesses réactionnelles observées.
\end{itemize}

\paragraph{Question 2}
\begin{itemize}
\item Dériver l'équation pour le coefficient de transfert de masse en partant de l'équation du flux d'un composant A dans le cadre d'une diffusion unidimensionnelle et en intégrant à travers la couche limite dans le cas ou A intervient dans une réaction $aA+bB+... \leftrightharpoons rR + sS+...$.
$$N_A=-C_t D_{Am} \frac{dy_A}{dz} + y_A (N_A + N_B + N_R + N_S+ ...)$$
\item 
\item Comment le coefficient de transfert de masse est-il calculé dans la pratique? Faire le lien avec la question 2.2 (10 min).
\end{itemize}

\paragraph{Question 3}
\begin{itemize}
\item Donner les principaux types de désactivation catalytiques et leurs caractéristiques majeures.
\end{itemize}

\paragraph{Question 4}
\begin{itemize}
\item Expliquer le "two-film theory" (5min,/5) 
\item Considérer avec plus de détail le cas particulier d'une réaction gaz-liquide simple, irréversible et instantanée. (15min,/15)
\end{itemize}

\paragraph{Exercice} (30 min, /20) [\textsf{A COMPLETER}] On considère un réacteur dans lequel se déroule une réaction $A \longrightarrow P$. On utilise un catalyseur sphérique qui a un diamètre de 1cm. En entrée du réacteur, on donne du A et un inerte qui sont 50/50\% en volume. En un certain point du réacteur, la fraction convertie en A est de 50\%. En ce même point règne une pression de $2\,bar$, une température de $300\,C^{\circ}$ et on observe une vitesse réactionnelle de $0,028\,[...]$. On sait que l'on peut calculer la vitesse intrinsèque par
$$ r = k C_A$$
et on a $$k = 10^4 \exp\Bigg( \frac{-55000}{RT} \Bigg)$$

Le mélange de A+P+inerte peut être considéré comme un mélange idéal de gaz parfaits.


\begin{itemize}
\item Donner la concentration de A \textbf{en phase gazeuse}.
\item Donner la température du catalyseur.
\item Donner le facteur d'efficacité. Quelle est la valeur de la vitesse \textbf{intrinsèque}?
\item Donner la valeur du module de Thiele. Justifier.
\item Donner la valeur de la diffusivité effective $D_{eA}$.
\end{itemize}

On avait aussi un petit carré avec diverses valeurs dont $\rho_s = 4 [...]$, $h_f=40[...]$, $a_m=50[...]$, $\Delta H=-200000[...]$.

\end{document}
