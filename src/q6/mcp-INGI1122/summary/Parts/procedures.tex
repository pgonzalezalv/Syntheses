\section{Procédures}
\subsection{Abstraction prodédurale}
L'abstraction procédurale consiste à \textbf{abstraire} un fragment de programme via un \textbf{nom}, une \textbf{interface} et une \textbf{spécification}. Ce fragment est utilisé pour son effet et indépendamment de son implémentation, comme un nouvel opérateur du langage.\\

$\bullet$\textbf{Abstraction} : L'abstraction par \textbf{paramétrage} généralise sur le traitement des données et ignore les données particulières. L'abstraction par \textbf{spécification} généralise sur l'effet du programme et ignore les implémentations particulières.\\
Le bénéfice de l'abstraction est la \textbf{modularité} qui est composée de la \textbf{localité} et de la \textbf{modifiabilité}. La localité permet de lire ou écrire l'implémentation d'une abstraction sans devoir consulter l'implémentation des abstractions qu'elle utilise. La modifiabilité permet de modifier l'implémentation d'une abstraction sans modifier l'implémentation des abstractions qui l'utilisent.
\subsubsection{Déclaration de procédure}
La déclaration d'une procédure est composée d'un \textbf{nom} $F$ (l'identifiant), des \textbf{paramètres} $V_1,...V_n$ (les variables), des \textbf{spécifications} $P_F,Q_F$ (les assertions) et du corps $S$ (l'instruction). Les spécifications ($P_F,Q_F$) respectent l'abstraction en termes de l'interface de $F$ avec ses paramètres et son résultat, contrairement aux variables locales dans $S$.\\

$\bullet$\textbf{Typage}: les types sont des spécifications supportées dans le langage. Le Type-checking vérifier que les variables et résultats sont du bon type.
\\

$\bullet$\textbf{Paramètres et résultats multiples}: en général, une procédure peut avoir plusieurs paramètres et retourner plusieurs résultats. Dans la suite, on ne présentera des procédures qu'avec un paramètre et ne retournant qu'un résultat. 
\\

$\bullet$\textbf{Instruction return}: \\Le résultat de la procédure $\equiv$ une variable distinguée result.\\
$\rightarrow$ return $E$; $\equiv$ result := $E;$\\
Axiome du return: \textbf{[Q[E/result]] return E [Q]} ou \textbf{wp(return E, Q) = Q[E/result]}
\\

Dans une déclaration de procédure, pour la preuve, il faut remplacer dans $S$ return E; par result:=E; et ensuite prouver $[P_F]S[Q_F]$
\subsection{Procédures pures}
$\bullet$\textbf{Appel de procédure}: de manière générale, elle se fait dans une expression.
\begin{lstlisting}
    y:= exp(x,10)/2;
    if exp(2,x)>1{x:=x-1;}
\end{lstlisting}
On décompose:
\begin{lstlisting}
    var u := exp(x,10); y:=u/2;
    var u := exp(2,x); if u>1 {x:=x-1;}
\end{lstlisting}
\textbf{u} est une variable fraîche, càd qui n'existe pas déjà dans le programme.\\

Une\textbf{ procédure pure} est une procédure qui ne modifie pas les variables non-locales de la procédure. La pré-condition $P_F$ porte uniquement sur les paramètres $V_1,...,V_n$ et la post-condition $Q_F$ porte uniquement sur les paramètres  $V_1,...,V_n$ non modifiées et le résultat $result$.\\

Pour une procédure pure $procedure F(V) \quad pre \quad P_F \quad post \quad Q_F \quad {S}$, si on a prouvé $[P_F] S [Q_F]$ alors on peut remplacer:\\
$U:=F(E);$\\
par : $assert \quad P_F[E/V]; \quad assume \quad Q_F[E/V,U/result];$\\

$\bullet$\textbf{Règle de la procédure pure}:\\

Pour une procédure $F(V) {S}$\\
Si $\quad  [P_F]S[Q_F]$\\
Alors $[P_F[E/V]]U:=F(E);[Q_F[E/V,U/result]]$\\

ou de manière équivalente\\

Pour une procédure $F(V) {S}$\\
Si $\quad  [P_F[V]]S[Q_F[V,result]]$\\
Alors $[P_F[E]]U:=F(E);[Q_F[E,U]]$\\
\subsection{Procédures avec effets}
Les procédures avec effets sont des procédures non-pures et donc peuvent modifient des variables non-locales, càd des paramètres $V$ transmis par référence ou des parties de variables.\\

$\bullet$ \textbf{Définition}: procedure F(V) pre $P_F$ post $Q_F$ modifies V \{S\}\\
La valeur de V est différente à l'entrée et à la sortie de S.\\

Pour une procédure avec effet $procedure F(V) \quad pre \quad P_F \quad post \quad Q_F \quad modifies \quad V \quad {S}$, si on a prouvé $[P_F] S [Q_F]$ alors on peut remplacer:\\
$U:=F(A);$ avec A une variable\\
par : $assert \quad P_F[E/V]; \quad A:= a_1; \quad assume \quad Q_F[E/V,U/result];$\\
avec $a_1 \equiv$ la valeur de A à la sortie de S.\\

$\bullet$\textbf{Règle de la procédure avec effet}:\\

Pour une procédure avec effet $F(V) {S}$\\
Si $\quad  [P_F]S[Q_F]$\\
Alors $[P[A/V]]U:=F(A);[Q[A/V,U/result]]$\\
C'est pareil que pour la procédure pure saud que A est une variable passée par référence dont la valeur diffère entre P et Q. \vfill

\subsection{Procédures récursives}
On suppose les appels corrects et on prouve le corps correct.
Pour prouver la terminaison, on définit un variant sur les paramètres de la procédure qui doit décroître pour les appels récursifs de F dans S.\\
procedure F(V) pre $P_F$ post $Q_F$ variant $Z_F$ \{S\}\\

$\bullet$\textbf{Règle de la procédure récursive}:\\
Pour une procédure $F(V) {S}$\\
Si $\quad$ si $ \quad [P_F[E/V] \wedge Z[E/V]<z_0] U :=F(E); [Q_F[E/V,U/result]] $\\ 
$.\quad \quad $alors $[P_F \wedge Z=z_0]S[Q_F]$\\
Alors $[P_F] S[Q_F]$\\
Et donc $[P_F[E/V]]U:=F(E);[Q_F[E/V,U/result]]$\\

Pour une procédure récursive $F(V)$ pre $P_F$ post $Q_F$ variant $Z_F$ ${S}$\\
Dans la preuve de $[P_F \wedge Z=z_0] S[Q_F]$, on peut remplacer les appels récursifs  $F(E)$ par\\  $assert \quad P_F[E/V] \wedge Z[E/V]<z_0; \quad assume \quad Q_F[E/V,U/result];$\\
