\documentclass[en]{../../../../../../epltest}

\usepackage{../../../../../../eplcode}

\hypertitle{Programming Languages Concepts}{6}{INGI}{1131}{2015}{March}{All}
{Tanguy De Bels\and Beno\^it Legat\and Florian Thuin}
{Peter Van Roy}

\lstset{language={Oz},morekeywords={for,do,lazy}}

\section{Digital logic simulation}
For the following circuit, do these two steps:
\begin{enumerate}
  \item Write the program to simulate this circuit according to the model used in the course.
  \item Write the equations for streams $A = a_0|a_1|\ldots$ and $B = b_0|b_1|\ldots$
    and solve for all elements of $B$ as fuctions of elements of $A$.
\end{enumerate}

\input{circuit_xor_delay}

\begin{solution}

\lstinputlisting{sol_q1.oz}

\[ B_{0} = 0 \] et \[ B_{1} = A_{0} \] car Delay ajoute juste 0

\[ B_{n} = B_{n-1} + A_{n-1} - 2 \times A_{n-1} \times B_{n-1} \]

avec :

\[ B_{n-1} = B_{n-2} + A_{n-2} - 2 \times A_{n-2} \times B_{n-2} \]

en exécutant cette définition récursive, on en revient au cas de base donc:

\[ B_{n} =  ( ...( ( A_{0} + A_{1} - 2 \times A_{0} \times A_{1} ) + A_{2} - 2 \times ( A_{0} + A_{1} - 2 \times A_{0} \times A_{1} ) \times A_{2} ) + A_{3} ...) + A_{n-1} - 2 \times ... \times A_{n-1}  \] ($B_{0}$ et $B_{1}$ ayant seulement une influence sur $B_{2}$ on commence avec la formule de $B_{2}$)
\end{solution}

\section{Execution tree}

For the following program, give the complete execution tree:
\begin{lstlisting}
thread A=1 end
thread B=A+2 end
thread C=A+B end
\end{lstlisting}
Remember to give the contents of memory as well as the threads at each step.
\begin{solution}
\input{sol_q2}
\end{solution}

\section{Definitions of concepts}
Give definitions of the following concepts or operations.
\begin{enumerate}
  \item Nondeterminism.
  \item Scheduler.
  \item WaitNeeded operation.
  \item Wait operation.
  \item Fairness.
\end{enumerate}

\begin{solution}
  See summary.
\end{solution}

\end{document}
