\documentclass[fr]{../../../../../../eplexam}

\hypertitle{Automatique linéaire}{6}{INMA}{1510}{2017}{Juin}{Mineure}
{Martin Braquet}
{Denis Dochain}

\section{}
    Soit la fonction de transfert $$\dfrac{K e^{-\theta s}}{1+s \tau}$$
    avec $K=4$, $\theta=0.5$ et $\tau = 3$.
    Analyser et comparer les méthodes Ziegler-Nichols et ITAE pour l'obtention des paramètres $K_p$ et $\tau_i$ d'un PI.
    
    Méthode de Ziegler-Nichols :
    
    $$K_p = 0.45K$$
    
    $$\tau_i = \dfrac{\tau}{1.2}$$
    
    Methode ITAE :
    
    $$K_p = \dfrac{0.586}{K}\left(\dfrac{\theta}{\tau}\right)^{-0.916}$$
    
    $$\tau_i = \dfrac{\tau}{1.03 -0.165\dfrac{\theta}{\tau}}$$

\section{}
    Que doit contenir au minimum le compensateur $C(s)$ pour annuler l'erreur statique lorsque $y^{*}(t)=\omega \, t^2$? Démontrez.

\begin{solution}

$$ C(s) = \dfrac{1}{s^{q+1}} \prod\limits_{k=1}^p \left(\dfrac{1}{s^2+w_k^2}\right)C_o(s)$$

L'erreur statique est notée $e_s$. La consigne par la transformée de Laplace devient  : $\dfrac{2w}{s^3}$. La tranformée de Laplace de la consigne sera notée $R(s)$.

$$e_s = r(\infty) - y(\infty) = e(\infty) = \lim_{t\to\infty} e(t) = \lim_{s\to 0} sE(s)$$

$$E(s) = \dfrac{R(s)}{1+C(s)G(s)} = \dfrac{2w}{s^3}\dfrac{1}{1+C(s)G(s)} $$

$$e_s = \lim_{s\to 0} s\dfrac{2w}{s^3}\dfrac{1}{1+C(s)G(s)} = \lim_{s\to 0}\dfrac{2w}{s^2}\dfrac{1}{1+C(s)G(s)} = \lim_{s\to 0} \dfrac{2w}{s^2C(s)G(s)} $$

Afin d'annuler l'erreur statique il nous faudra un compensateur C(s) de troisième ordre:

$$C(s) = \dfrac{C_o(s)}{s^3}$$
$$e_s = \lim_{s\to 0}\dfrac{2w}{\dfrac{C_o(s)}{s}} = \lim_{s\to 0}\dfrac{2w}{C_o(s)}s = 0$$

En regardant la formule générale ceci est correct ($q=2$).
    
\end{solution}

\section{}
On a la fonction de transfert $G(s) = \dfrac{2}{(s + 1) (3s - 1)} $.

Analyser la stabilité pour ces différents contrôleurs:
\begin{enumerate}
    \item P
    \item PI
    \item PD
    \item PID
\end{enumerate}

\begin{solution}

Relations générales:
\[
 C(s)=\frac{N_C(s)}{D_C(s)} \qquad G(s)=\frac{N_G(s)}{D_G(s)}
\]

\[
 Tr(s)=\frac{C(s)G(s)}{1+C(s)G(s)}=\frac{N_C(s)N_D(s)}{N_C(s)N_D(s)+D_C(s)D_D(s)}=\frac{num(s)}{den(s)}
\]

\begin{enumerate}
    \item P: $C(s) = k_p $
    
    $$den(s)= 3s^2+2s+(2k_p-1)$$
    Condition nécessaire et suffisante pour une equation de seconde degré : $a_i>0$, donc c'est stable si $k_p>1/2$.

    \item PI: $C(s) = k_p\left(1+\dfrac{1}{\tau_is}\right)$
    \[
     den(s)=3\tau_is^3 +2\tau_is^2 +(2k_p-1)\tau_is+2k_p=3\tau_i\left(s^3+\dfrac{2}{3}s^2 + \dfrac{2k_p-1}{3}s+\dfrac{2k_p}{3\tau_i}\right)
    \]
    On travaille avec la matrice de Hurwitz, on doit respecter les inéquations suivantes: $\Delta_1>0$  $\Delta_2>0$ $\Delta_3>0$.

    Après résolution on obtient

    $$(2k_p-1)\tau_i-3k_p>0 \qquad \tau_i>0 \qquad k_p>0 $$
    On a donc $\tau_i > \frac{3K_p}{2K_p-1}$ et $Kp>\frac{1}{2}$.
    
    \item PD: $C(s) = k_p(1+\tau_ds)$
    \[
     den(s)=3s^2+2(k_p\tau_d+1)s+(2k_p-1)
    \]
    $$2(k_p\tau_d+1)>0 \qquad 2k_p-1>0$$

    Stable si $k_p>1/2$ et $\tau_d>-1/k_p$.
    
    \item PID: $C(s) = kp\frac{\tau_i\tau_ds^2+\tau_is+1}{s\tau_i}$
    \[
     den(s)=3\tau_is^3 +2(k_p\tau_d+1)\tau_is^2 +(2k_p-1)\tau_is+2k
    \]
    Par Hurwitz, on a
    $$\tau_i>\frac{3k_p}{(1+k_p\tau_d)(2k_p-1)}\qquad \tau_i(1+k_p\tau_d)>0 \qquad k_p>0$$
    Par exemple, le choix $k_p=1$, $\tau_d=1$ et $\tau_i=5$ stabilise le système en boucle fermée.
\end{enumerate}

\end{solution}

\section{}
    Soient les fonctions de transfert $G(s) = \dfrac{3}{s^2 + 4s + 4}$ et $H(s) = \dfrac{2}{s+7}$.\\
    \begin{enumerate}
        \item Trouver un compensateur qui permet de n'avoir que deux pôles, en -1 et -3.
        \item Implémenter une structure de contrôle permettant d'éliminer l'effet de $v(t)$. Quel est son nom?
        \item Idem 1. et 2. pour $G(s) = \dfrac{3 (1-s)}{s^2 + 4s + 4}$ et $H(s) = \dfrac{1}{s+6}$.
    \end{enumerate}
    
\begin{solution}

\begin{enumerate}
\item 
    On essaye de résoudre le problème avec un compensateur proportionnel: $C(s) = k_p$. Le dénominateur avec un pôle $-1$ et un pôle $-3$ est égal à une équation d'ordre 2 : $s^2+4s+3$. 

    $$Tr(s) = \dfrac{C(s)G(s)}{1+C(s)G(s)} = \dfrac{3k_p}{s^2+4s+3k_p+4}$$
    On que $3k_p+4=3$,

    $$k_p=-\dfrac{1}{3}$$

 \item
    On utilise une commande anticipatrice: $U(s)=[R(s)-Y(s)]C_1(s)-V(s)C_2(s)$.
    
    On veut que $Y(s)$ ne dépende plus de $V(s)$: $Y(s)=\Omega(s)R(s)$ avec $\Omega(s)$ une fonction de transfert que l'on détermine. En manipulant la relation, 
    \[
     Y(s)=\frac{\Omega(s)}{1-\Omega(s)}[R(s)-Y(s)]
    \]
    
    On sait aussi que $Y(s)=U(s)G(s)+V(s)H(s)$, donc 
    \[
     U(s)=\frac{Y(s)}{G(s)}-\frac{V(s)H(s)}{G(s)}=\frac{Y(s)}{G(s)}-\frac{H(s)}{G(s)}V(s)=\underbrace{\frac{1}{G(s)}\frac{\Omega(s)}{1-\Omega(s)}}_{C_1(s)}[R(s)-Y(s)]-\underbrace{\frac{H(s)}{G(s)}}_{C_2(s)}V(s)
    \]
    En choissisant ainsi ces valeurs pour $C_1(s)$ et $C_2(s)$, on annule totalement l'influence de la perturbation pour toute valeur de $v(t)$.

    \[
     C_1(s)=\frac{1}{G(s)}\frac{\Omega(s)}{1-\Omega(s)}=\frac{s^2 + 4s + 4}{3}\frac{\Omega(s)}{1-\Omega(s)}
    \]
    \[
     C_2(s)=\frac{H(s)}{G(s)}=\frac{(s^2 + 4s + 4)(s+7)}{6}
    \]

 \item 
    Le second système à contrôler est plus complexe, une simple action integrale ne marche pas. On utilise donc un compensateur PD,$$C(s) = k_p(1+\tau_ds)$$ qui augmente les paramètres manipulables et maintient l'ordre du dénominateur à 2 (plus facile pour la résolution).

    $$Tr(s) = \dfrac{3k_p(1-s)(\tau_ds+1)}{(1-3k_p\tau_d)s^2+\left[3k_p(\tau_d-1)+4\right]s+(3k_p+4)}$$

    Le problème se traduit en un système d'équations à 2 inconnues :

    \begin{equation*}
            \left \{
            \begin{array}{l @{} l}
                \dfrac{3k_p(\tau_d-1)+4}{1-3k_p\tau_d} = 4 \\
                \dfrac{3k_p+4}{1-3k_p\tau_d}= 3\\
            \end{array}
            \right.
            \qquad\Longrightarrow\qquad
            \left \{
            \begin{array}{l @{} l}
                k_p = -\dfrac{5}{24}\\
                \tau_d = \dfrac{1}{5}\\
            \end{array}
            \right.
    \end{equation*}
    A propos de la commande anticipatrice,
     \[
     C_1(s)=\frac{1}{G(s)}\frac{\Omega(s)}{1-\Omega(s)}=\frac{s^2 + 4s + 4}{3(1-s)}\frac{\Omega(s)}{1-\Omega(s)}
    \]
    Puisque $G(s)$ est à non minimum de phase, le filtre $C_1(s)$ a un pôle instable, il est donc impossible d'annuler la perturbation par une commande prédictive.
    
\end{enumerate}

\end{solution}

\section{}
Soit le système représentant un réacteur chimique suivant :
\begin{equation*}
    \left \{
    \begin{array}{l @{} l}
        \dot{x}_1 = d(x_{1,in} - x_1) - k_0 \,x_1\\
        \dot{x}_2 = -d\, x_2 + b \,k_0\, x_1\\
    \end{array}
    \right.
\end{equation*}
Avec $x_1$ la concentration en réactifs, $x_2$ la concentration en produits, $d$ le facteur de dilution et $k_0$ la constante d'équilibre.
    
On s'intéresse à deux observateurs d'état différents :
\begin{itemize}
     \item La concentration en réactifs est mesurée en temps réel ($y=x_1$).
    $k_1$ et $k_2$ sont choisis par l'utilisateur.
    \begin{equation*}
        \left \{
        \begin{array}{l @{} l}
            \dot{\hat{x}}_1 = d(x_{1,in} - \hat{x}_1) - k_0\, \hat{x}_1 + k_1 (x_{1} - \hat{x}_1)\\
            \dot{\hat{x}}_2 = -d \,\hat{x}_2 + b \,k_0\, \hat{x}_1 - k_2 (x_{1} - \hat{x}_1)\\
        \end{array}
        \right.
    \end{equation*}
    
    \item La concentration en produits est mesurée en temps réel ($y=x_2$).
    $k_3$ et $k_4$ sont choisis par l'utilisateur.
    \begin{equation*}
        \left \{
        \begin{array}{l @{} l}
            \dot{\hat{x}}_1 = d(x_{1,in} - \hat{x}_1) - k_0 \,\hat{x}_1 + k_3 (x_{2} - \hat{x}_2)\\
            \dot{\hat{x}}_2 = -d\, \hat{x}_2 + b\, k_0\, \hat{x}_1 + k_4 (x_{2} - \hat{x}_2)\\
        \end{array}
        \right.
    \end{equation*}
    
\end{itemize}
    
\begin{enumerate}
    \item Étudiez l'observabilité du système dans les deux cas.
    \item Comment l'observabilité/la non-observabilité de ces systèmes influence leur dynamique? Plus précisément, intéressez vous à la dynamique des erreurs d'observation $e_1 = x_1-\hat{x}_1$ et $e_2 = x_2-\hat{x}_2$.
\end{enumerate}

\section{}
On a la fonction de transfert $G(s) = \dfrac{5 e^{-2 s}}{0.25 s + 1} $.

On souhaite n'avoir qu'un seul pôle en -5.
\begin{enumerate}
    \item Quel contrôleur $C(s)$ faut-il employer? À quoi faut-il faire attention?
    \item En utilisant la même forme de compensateur qu'au point 1, synthétiser un loi de commande qui permette d'éliminer les effets négatifs sur le comportement en boucle fermée. Comment se nomme cette loi de commande? Quelles précautions devez-vous prendre?
    \item Analyser la stabilité en boucle fermée pour les points 1 et 2. Quelles sont vos conclusions?
    \item Pour la deuxième configuration, vous avez fait la synthèse en pensant que le retard était 1 (alors que sa vraie valeur est bien 2). Analysez la stabilité du système en boucle fermée dans ce cas de figure.
\end{enumerate}

Si, à un moment donné de votre démarche, vous devez approximer le retard, utilisez l'approximation de Taylor. 

Justifiez vos réponses.
    
\begin{solution}

\begin{enumerate}
 \item 
    On veut $$Tr(s)=\frac{e^{-2s}}{1+0.2s}$$ 
    Donc $C(s)=\frac{1}{44}\left(1+\frac{4}{s}\right)$.
 
 \item 
    Dans ce paragraphe $G(s)$ est la fonction qui décrit le système sans considèrer le retard $e^{-\theta s}$.
    Une structure de contrôle qui annule les effets du retard n'existe pas. Mais on peut empécher que ce retard sur $G(s)$ ne déstabilise le système grâce à un prédicteur de Smith. Le prédicteur de Smith permet de transférer le retard de $G(s)$ vers la fonction de tranfert $Tr(s)$. Cette structure de contrôle est composée d'un compensateur et un boucle de rétroaction négative non-unitaire de la forme $G(s)(1-e^{-\theta s})$. Le paramètre $\theta$ (le retard) doit être connu à priori sinon le prédicteur de Smith n'est pas optimal. Voici la fonction de transfert d'un prédicteur de Smith:
    $$ C_{Smith}(s) = \dfrac{C(s)}{1+(1-e^{-\theta s})C(s)G(s)}$$
    $$Tr(s) = \dfrac{C_{Smith}(s)G(s)e^{-\theta s}}{1+C_{Smith}(s)G(s)e^{-\theta s}} = \dfrac{C(s)G(s)}{1+C(s)G(s)}e^{-\theta s}$$

    Donc $C(s)=\frac{1}{4}\left(1+\frac{4}{s}\right)$
  
 \item Les 2 systèmes sont stables, tant que l'approximation de Taylor reste valable.
   
 \item C'est pire en sous-évaluant le retard, $Tr(s)$ possède un pôle instable en $p=5/4$ lorsque le retard est sous-évalué à 1.  
\end{enumerate}

\end{solution}

\end{document}
